% Licensed to the Apache Software Foundation (ASF) under one
% or more contributor license agreements.  See the NOTICE file
% distributed with this work for additional information
% regarding copyright ownership.  The ASF licenses this file
% to you under the Apache License, Version 2.0 (the
% "License"); you may not use this file except in compliance
% with the License.  You may obtain a copy of the License at

%   http://www.apache.org/licenses/LICENSE-2.0

% Unless required by applicable law or agreed to in writing,
% software distributed under the License is distributed on an
% "AS IS" BASIS, WITHOUT WARRANTIES OR CONDITIONS OF ANY
% KIND, either express or implied.  See the License for the
% specific language governing permissions and limitations
% under the License.

% When using TeXShop on the Mac, let it know the root document.
% The following must be one of the first 20 lines.
% !TEX root = ../design.tex

\chapter{Neural Network}

\begin{moduleinfo}
\item[Authors] {Xixuan Feng, Cooper Sloan}
\end{moduleinfo}

% Abstract. What is the problem we want to solve?
This module implements artificial neural network \cite{ann_wiki}.

\section{Multilayer Perceptron}
Multilayer perceptron is arguably the most popular model among many neural network models \cite{mlp_wiki}.
Here, we learn the coefficients by minimizing a least square objective function, or cross entropy (\cite{bertsekas1999nonlinear}, example 1.5.3).
The parallel architecture is based on the paper by Zhiheng Huang \cite{mlp_parallel}.

% Background. Why can we solve the problem with gradient-based methods?
\subsection{Solving as a Convex Program}
Although the objective function is not necessarily convex, gradient descent or incremental gradient descent are still commonly-used algorithms to learn the coefficients.
To clarify, gradient-based methods are not different from the famous backpropagation, which is essentially a way to compute the gradient value.

\subsection{Formal Description}
Having the convex programming framework, the main tasks of implementing a learner include:
(a) choose a subset of algorithms;
(b) implement the related computation logic for the objective function, e.g., gradient.

For multilayer perceptron, we choose incremental gradient descent (IGD).
In the remaining part of this section, we will give a formal description of the derivation of objective function and its gradient.

\paragraph{Objective function.}
We mostly follow the notations in example 1.5.3 from Bertsekas \cite{bertsekas1999nonlinear}, for a multilayer perceptron that has $N$ layers (stages), and the $k^{th}$ stage has $n_k$ activation units ($\phi : \mathbb{R} \to \mathbb{R}$), the objective function for regression is given as
\[f_{(x, y)}(u) = \frac{1}{2} \|h(u, x) - y\|_2^2,\]
and for classification the objective function is given as
\[f_{(x, y)}(u) = \sum_i (\log(h_i(u, x)) * z_i + (1-\log(h_i(u, x))) *( 1- z_i) ,\]
where $x \in \mathbb{R}^{n_0}$ is the input vector, $y \in \mathbb{R}^{n_N}$ is the output vector (one hot encoded for classification),~\footnote{Of course, the objective function can be defined over a set of input-output vector pairs, which is simply given as the addition of the above $f$.}
and the coefficients are given as
\[u = \{ u_{k-1}^{sj} \; | \; k = 1,...,N, \: s = 0,...,n_{k-1}, \: j = 1,...,n_k\},\]
And are initialized from a uniform distribution as follows:
\[u_{k}^{sj} = uniform(-r,r),\]
where r is defined as follows:
\[r = \sqrt{\frac{6}{n_k+n_{k+1}}}\]
With regularization, an additional term enters the objective function, given as
\[\sum_{u_k^{sj}} \frac{1}{2} \lambda u_k^{sj2} \]
This still leaves $h : \mathbb{R}^{n_0} \to \mathbb{R}^{n_N}$ as an open item.
Let $o_k \in \mathbb{R}^{n_k}, k = 1,...,N$ be the output vector of the $k^{th}$ layer. Then we define $h(u, x) = o_N$, based on setting $o_0 = x$ and the $j^{th}$ component of $o_k$ is given in an iterative fashion as~\footnote{$o_k^0 \equiv 1$ is used to simplified the notations, and $o_k^0$ is not a component of $o_k$, for any $k = 0,...,N$.}
\[\begin{alignedat}{5}
    o_k^j = \phi \left( \sum_{s=0}^{n_{k-1}} o_{k-1}^s u_{k-1}^{sj} \right), &\quad k = 1,...,N, \; j = 1,...,n_k
\end{alignedat}\]

\paragraph{Gradient of the End Layer.}
Let's first handle $u_{N-1}^{st}, s = 0,...,n_{N-1}, t = 1,...,n_N$.
Let $y^t$ denote the $t^{th}$ component of $y \in \mathbb{R}^{n_N}$, and $h^t$ the $t^{th}$ component of output of $h$.
\[\begin{aligned}
    \frac{\partial f}{\partial u_{N-1}^{st}}
    &= \left( h^t(u, x) - y^t \right) \cdot \frac{\partial h^t(u, x)}{\partial u_{N-1}^{st}} \\
    &= \left( o_N^t - y^t \right) \cdot \frac{\partial o_N^t}{\partial u_{N-1}^{st}} \\
    &= \left( o_N^t - y^t \right) \cdot \frac{\partial \phi \left( \sum_{s=0}^{n_{N-1}} o_{N-1}^s u_{N-1}^{st} \right)}{\partial u_{N-1}^{st}} \\
    &= \left( o_N^t - y^t \right) \cdot \phi' \left( \sum_{s=0}^{n_{N-1}} o_{N-1}^s u_{N-1}^{st} \right) \cdot o_{N-1}^s \\
\end{aligned}\]
To ease the notation, let the input vector of the $j^{th}$ activation unit of the $(k+1)^{th}$ layer be
\[\mathit{net}_k^j =\sum_{s=0}^{n_{k-1}} o_{k-1}^s u_{k-1}^{sj},\]
where $k = 1,...,N, \; j = 1,...,n_k$, and note that $o_k^j =\phi(\mathit{net}_k^j)$. Finally, the gradient
\[\frac{\partial f}{\partial u_{N-1}^{st}} = \left( o_N^t - y^t \right) \cdot \phi' ( \mathit{net}_N^t ) \cdot o_{N-1}^s\]
For any $s = 0,...,n_{N-1}, t =1,...,n_N$, we are given $y^t$, and $o_N^t, \mathit{net}_N^t, o_{N-1}^s$ can be computed by forward iterating the network layer by layer (also called the feed-forward pass). Therefore, we now know how to compute the coefficients for the end layer $u_{N-1}^{st}, s = 0,...,n_{N-1}, t =1,...,n_N$.

\subsubsection{Backpropagation}
For inner (hidden) layers, it is more difficult to compute the partial derivative over the input of activation units (i.e., $\mathit{net}_k, k = 1,...,N-1$).
That said, $\frac{\partial f}{\partial \mathit{net}_N^t} = (o_N^t - y^t) \phi'(\mathit{net}_N^t)$ is easy, where $t = 1,...,n_N$, but $\frac{\partial f}{\partial \mathit{net}_k^j}$ is hard, where $k = 1,...,N-1, j = 1,..,n_k$.
This hard-to-compute statistic is referred to as \textit{delta error}, and let $\delta_k^j = \frac{\partial f}{\partial \mathit{net}_k^j}$, where $k = 1,...,N-1, j = 1,..,n_k$.
If this is solved, the gradient can be easily computed as follow
\[\frac{\partial f}{\partial u_{k-1}^{sj}} = \boxed{\frac{\partial f}{\partial \mathit{net}_k^j}} \cdot \frac{\partial \mathit{net}_k^j}{\partial u_{k-1}^{sj}} = \boxed{\delta_k^j} o_{k-1}^s,\]
where $k = 1,...,N-1, s = 0,...,n_{k-1}, j = 1,..,n_k$.
To solve this, we introduce the popular backpropagation below.

\paragraph{Error Back Propagation.}
Since we know how to compute $\delta_N^t, t = 1,...,n_N$, we try to compute $\delta_{k}^j, j = 1,...,n_{k}$, given $\delta_{k+1}^t, t = 1,...,n_{k+1}$, for any $k = 1,...,N-1$.
First,
\[
    \delta_{k}^j
    = \frac{\partial f}{\partial \mathit{net}_{k}^j}
    = \frac{\partial f}{\partial o_{k}^j} \cdot \frac{\partial o_{k}^j}{\partial \mathit{net}_{k}^j}
    = \frac{\partial f}{\partial o_{k}^j} \cdot \phi'(\mathit{net}_{k}^j)
\]

\[\begin{alignedat}{5}
    \frac{\partial f}{\partial o_{k}^j} = \sum_{t=1}^{n_{k+1}} \left( \frac{\partial f}{\partial \mathit{net}_{k+1}^t} \cdot \frac{\partial \mathit{net}_{k+1}^t}{\partial o_{k}^j} \right),
    &\quad k = 1,...,N-1, \: j = 1,...,n_{k}
\end{alignedat}\]
Using the above equation, we can solve delta error backward iteratively
\[\begin{aligned}
    \delta_{k}^j
    &= \frac{\partial f}{\partial o_{k}^j} \cdot \phi'(\mathit{net}_{k}^j) \\
    &= \sum_{t=1}^{n_{k+1}} \left( \frac{\partial f}{\partial \mathit{net}_{k+1}^t} \cdot \frac{\partial \mathit{net}_{k+1}^t}{\partial o_{k}^j} \right) \cdot \phi'(\mathit{net}_{k}^j) \\
    &= \sum_{t=1}^{n_{k+1}} \left( \delta_{k+1}^t \cdot \frac{\partial \left( \sum_{s=0}^{n_{k}} o_{k}^s u_{k}^{st} \right) }{\partial o_{k}^j} \right) \cdot \phi'(\mathit{net}_{k}^j) \\
    &= \sum_{t=1}^{n_{k+1}} \left( \delta_{k+1}^t \cdot u_{k}^{jt} \right) \cdot \phi'(\mathit{net}_{k}^j) \\
\end{aligned}\]
To sum up, we need the following equation for error back propagation
\[\boxed{\delta_{k}^j = \sum_{t=1}^{n_{k+1}} \left( \delta_{k+1}^t \cdot u_{k}^{jt} \right) \cdot \phi'(\mathit{net}_{k}^j)}\]
where $k = 1,...,N-1$, and $j = 1,...,n_{k}$.

\subsubsection{The $\mathit{Gradient}$ Function}
\begin{algorithm}[mlp-gradient$(u, x, y)$] \label{alg:mlp-gradient}
\alginput{Coefficients $u = \{ u_{k-1}^{sj} \; | \; k = 1,...,N, \: s = 0,...,n_{k-1}, \: j = 1,...,n_k\}$,\\
start vector $x \in \mathbb{R}^{n_0}$,\\
end vector $y \in \mathbb{R}^{n_N}$,\\
activation unit $\phi : \mathbb{R} \to \mathbb{R}$}
\algoutput{Gradient value $\nabla f(u)$ that consists of components $\nabla f(u)_{k-1}^{sj} = \frac{\partial f}{\partial u_{k-1}^{sj}}$}
\begin{algorithmic}[1]
    \State $(\mathit{net}, o) \set$ \texttt{feed-forward}$(u, x, \phi)$
    \State $\delta_N \set$ \texttt{end-layer-delta-error}$(\mathit{net}, o, y, \phi')$
    \State $\delta \set$ \texttt{back-propogate}$(\mathit{net}, o, y, u, \phi')$
    \For{$k = 1,...,N$}
        \For{$s = 0,...,n_{k-1}$}
            \For{$j = 1,...,n_k$}
                \State $\nabla f(u)_{k-1}^{sj} \set \delta_k^j o_{k-1}^s$
                \Comment{Can be put together with the computation of delta $\delta$}
            \EndFor
        \EndFor
    \EndFor
    \State \Return $\nabla f(u)$
\end{algorithmic}
\end{algorithm}

\paragraph{Activation Units $\phi$.}
Common examples of activation units are
\[\begin{alignedat}{3}
\phi(\xi) &= \frac{1}{1 + e^{-\xi}}, &\quad \text{ (logistic function),}\\
\phi(\xi) &= \frac{e^{\xi} - e^{-\xi}}{e^{\xi} + e^{-\xi}}, &\quad \text{ (hyperbolic tangent function)}\\
\phi(\xi) &= max(x,0), &\quad \text{ (rectified linear function)}\\
\end{alignedat}\]

\begin{algorithm}[feed-forward$(u, x, \phi)$] \label{alg:feed-forward}
\alginput{Coefficients $u = \{ u_{k-1}^{sj} \; | \; k = 1,...,N, \: s = 0,...,n_{k-1}, \: j = 1,...,n_k\}$,\\
input vector $x \in \mathbb{R}^{n_0}$,\\
activation unit $\phi : \mathbb{R} \to \mathbb{R}$}
\algoutput{Input vectors $\mathit{net} = \{\mathit{net}_k^j \; | \; k = 1,...,N, \: j = 1,...,n_k\}$,\\
output vectors $o = \{o_k^j \; | \; k = 0,...,N, \: j = 0,...,n_k\}$}
\begin{algorithmic}[1]
    \For{$k = 0,...,N$}
        \State $o_k^0 \set 1$
    \EndFor
    \State $o_0 \set x$ \Comment{For all components $o_0^j, x^j, \; j = 1,...,n_0$}
    \For{$k = 1,...,N$}
        \For{$j = 1,...,n_k$}
            \State $\mathit{net}_k^j \set 0$
            \For{$s = 0,...,n_{k-1}$}
                \State $\mathit{net}_k^j \set \mathit{net}_k^j + o_{k-1}^s u_{k-1}^{sj}$
            \EndFor
            \State $o_k^j = \phi(\mathit{net}_k^j)$ \Comment{Where the activation function for the final layer is identity for regression and softmax for classification.}
        \EndFor
    \EndFor
    \State \Return $(\mathit{net}, o)$
\end{algorithmic}
\end{algorithm}

\begin{algorithm}[back-propogate$(\delta_N, \mathit{net}, u, \phi')$] \label{alg:back-propogate}
\alginput{input vectors $\mathit{net} = \{\mathit{net}_k^j \; | \; k = 1,...,N, \: j = 1,...,n_k\}$,\\
output vectors $o = \{o_k^j \; | \; k = 0,...,N, \: j = 0,...,n_k\}$,\\
end vector $y \in \mathbb{R}^{n_N}$,\\
coefficients $u = \{ u_{k-1}^{sj} \; | \; k = 1,...,N, \: s = 0,...,n_{k-1}, \: j = 1,...,n_k\}$,\\
derivative of activation unit $\phi' : \mathbb{R} \to \mathbb{R}$}
\algoutput{Delta $\delta = \{\delta_k^j \; | \; k = 1,...,N, \: j = 1,...,n_k\}$}
\begin{algorithmic}[1]
    \For{$t = 1,...,n_N$}
            \State $\delta_N^t \set (o_N^t - y^t)$ \Comment{This applies for identity activation and mean square error loss and softmax activation with cross entropy loss}
    \EndFor
    \For{$k = N-1,...,1$}
        \For{$j = 0,...,n_k$}
            \State $\delta_k^j \set 0$
            \For{$t = 1,...,n_{k+1}$}
                \State $\delta_k^j \set \delta_k^j + \delta_{k+1}^t u_k^{jt}$
            \EndFor
            \State $\delta_k^j \set \delta_k^j \phi'(\mathit{net}_k^j)$
        \EndFor
    \EndFor
    \State \Return $\delta$
\end{algorithmic}
\end{algorithm}

\begin{algorithm}[mlp-train-iteration$(X, Y, \eta)$] \label{alg:mlp-train-iteration}
\alginput{
start vectors $X_{i...m} \in \mathbb{R}^{n_0}$,\\
end vectors $Y_{i...m} \in \mathbb{R}^{n_N}$,\\
learning rate $\eta$,\\}
\algoutput{Coefficients $u = \{ u_{k-1}^{sj} \; | \; k = 1,...,N, \: s = 0,...,n_{k-1}, \: j = 1,...,n_k\}$}
\begin{algorithmic}[1]
    \State \texttt{Randomnly initialize u}
    \For{$i = 1,...,m$}
        \State $\nabla f(u) \set \texttt{mlp-gradient}(u,X_i,Y_i)$
        \State $u \set u - (\eta \nabla f(u) u + \lambda u)$
    \EndFor
    \State \Return $u$
\end{algorithmic}
\end{algorithm}

\begin{algorithm}[mlp-train-parallel$(X, Y, \eta, s, t)$] \label{alg:mlp-train-parallel}
\alginput{
start vectors $X_{i...m} \in \mathbb{R}^{n_0}$,\\
end vectors $Y_{i...m} \in \mathbb{R}^{n_N}$,\\
learning rate $\eta$,\\
segments $s$,\\
iterations $t$,\\}
\algoutput{Coefficients $u = \{ u_{k-1}^{sj} \; | \; k = 1,...,N, \: s = 0,...,n_{k-1}, \: j = 1,...,n_k\}$}
\begin{algorithmic}[1]
    \State \texttt{Randomnly initialize u}
    \For{$j = 1,...,s$}
        \State $X_j \set \texttt{subset-of-X}$
        \State $Y_j \set \texttt{subset-of-Y}$
    \EndFor
    \For{$i = 1,...,t$}
        \For{$j = 1,...,s$}
            \State $u_j \set copy(u)$
            \State $u_j \set \texttt{mlp-train-iteration}(X_j, Y_j, \eta)$
        \EndFor
        \State $u \set \texttt{weighted-avg}(u_{1...s})$
    \EndFor
    \State \Return $u$
\end{algorithmic}
\end{algorithm}
